\chapter{Proposta}
\section{Contexto}
\label{pr:contex}

A Arquitetura Orientada a Serviços, do inglês \textit{Service-Oriented Architecture}, ou simplesmente SOA, tem surgido ao longo dos últimos anos como uma das abordagens preferidas de design de arquitetura para construção de sistemas~\cite{erl2008soa}.

Com objetivos alinhados aos objetivos do paradigma SOC \textit{(Service-Oriented Computing)}, SOA possibilita a criação de novas aplicações com maior coerência, rapidez e diminuição nos custos, tudo isso com excelente aproveitamento do legado~\cite{erl2008soa}.

Baseada em padrões abertos e na visão onipresente da internet, SOA tem como princípio fundamental a idéia de serviços como unidades que representam módulos do negócio ou funcionalidades de aplicação~\cite{erl2008soa}~\cite{imb2007soa}.

Na visão de SOA, um serviço é um componente que atende a uma função de negócio. Ele pode responder a requisições ocultando os detalhes de sua implementação, e é descrito através de contratos que expressam seu objetivo e suas capacidades.

Buscando maior agilidade, flexibilidade e reuso de componentes pré-existentes no desenvolvimento de aplicações, diversas empresas de tecnologia da informação vêm adotando SOA na construção de seus sistemas, quebrando assim, paradigmas das antigas metodologias de desenvolvimento para arquiteturas monolíticas.

Nesse contexto, temos a plataforma de serviços OSGi, que é um conjunto de especificações definidas pela OSGi Alliance. OSGi provê suporte à modelagem e desenvolvimento de sistemas modulares na linguagem Java~\cite{hall2010osgi}. A idéia básica é resolver o problema de criar softwares monolíticos, ou seja, softwares projetados sem modularidade, facilitando a reutilização e manutenção de  componentes, o que torna a solução robusta, barata e confiável~\cite{davis2009open}.

OSGi introduz um modelo de programação orientada a serviço, o que alguns autores apresentam como \textit{SOA in a Virtual Machine}~\cite{hall2010osgi}, separando de fato a interface da implementação. Isso mostra um grande potencial para a construção de aplicações orientadas à serviço.


Outra tecnologia que compartilha vários princípios de SOC é a tecnologia de Web Services. Web Services fornecem uma forma padrão de interoperabilidade entre diferentes aplicações, rodando em uma variedade de plataformas e frameworks.~\cite{w3c2002ws}


Assim, em uma arquitetura orientada a serviços, módulos podem consumir serviços providos por tecnologias diferentes, apenas identificando o serviço requerido e realizando o \textit{binding}. Estabelecendo assim um contrato entre o consumidor e o provedor do serviço~\cite{oracle2005ws}.  

Porém, como garantir que os contratos a nível de serviço entre o consumidor e provedor são realmente respeitados? Ou quando for caracterizada uma quebra de contrato, onde o provedor não cumpre com suas responsabilidades, selecionar dinamicamente um serviço equivalente que realmente cumpra com os requisitos desejados?

É pensando nessas perguntas, que vemos no monitoramento de serviço conjugado com a análise dos dados monitorados, uma possível solução para esses problemas.

\newpage
\section{Objetivos}
\label{pr:objectives}

Este trabalho de graduação tem como principal objeto o desenvolvimento de um mecanismo de monitoramento e seleção de serviços, capaz de capturar e armazenar dados relativos as métricas de qualidade de serviço monitoradas, e em posse desses dados, realizar a seleção do serviço requerido.

As métricas de QoS monitoradas são requeridas pelo consumidor do serviço, através da definição de contratos de serviço junto ao provedor. A partir da análise desses contratos em relação aos dados obtidos do monitoramento, o provedor mais adequado será escolhido dentre os possíveis candidatos.

\newpage
\chapter{Cronograma}
\label{pr:chrono}

\section{Programaç\~ao das Atividades}
{%
\begin{center}
\begin{table*}[h]
\begin{supertabular}[]{|l|c|c|c|c|}\hline
\textbf{Atividade} & \textbf{Setembro} & \textbf{Outubro} & \textbf{Novembro} & \textbf{Dezembro}\\\hline
Estudar SOA, OSGi e iPojo& \textbf{×} & \textbf{×} &   & \textbf{×}\\\hline
Implementar Interceptadores & \textbf{×} & \textbf{×} &  & \\\hline
Estudar DOSGi &   & \textbf{×} &   & \textbf{×}\\\hline
Implementar Seleção de Serviços &   & \textbf{×} & \textbf{×} &  \\\hline
Elaborar Monografia &   &   & \textbf{×} & \textbf{×}\\\hline
\end{supertabular}
\caption{Tabela das Atividades Programadas}
\end{table*}
\end{center}
}%

