\chapter{Proposta}
\section{Contexto}
\label{pr:contex}

A arquitetura Orientada à Serviços, do inglês \textit{Service-Oriented Architecture}, ou simplesmente SOA, tem surgido ao longo dos últimos anos como uma das abordagens preferidas de design arquitetura, modelo de desenvolvimento e integração de software.

Baseada em padrões abertos e na visão onipresente da internet, SOA tem como princípio fundamental a idéia de serviços como unidades que representam módulos do negócio ou funcionalidades de aplicação.
\\

Visando maior agilidade na composição de soluções, flexibilidade na implantação e reuso de componentes pré-existentes, com maior grau de confiabilidade, que consequentemente gera redução de custos, diversas empresas de tecnologia da informação vêm adotando SOA na construção de suas soluções, quebrando paradigmas das antigas metodologias de desenvolvimento monolíticas.



\newpage
\section{Objetivos}
\label{pr:objectives}

Este trabalho de graduação tem como principal objeto o desenvolvimento de uma arquitetura de monitoramento e seleção de serviços para a plataforma OSGi, capaz de capturar e armazenar dados relativos as métricas de qualidade de serviço monitoradas e em posse desses dados, realizar a seleção do serviço.
\\

As métricas de QoS monitoradas são requeridas pelo consumidor do serviço, através da definição de contratos de serviço junto ao provedor. A partir da análise desses contratos em relação aos dados obtidos do monitoramento, o provedor mais adequado será escolhido dentre os possíveis candidatos.

\newpage
\chapter{Cronograma}
\label{pr:chrono}

\section{Programaç\~ao das Atividades}
{%
\begin{center}
\begin{table*}[h]
\begin{supertabular}[]{|l|c|c|c|c|}\hline
\textbf{Atividade} & \textbf{Setembro} & \textbf{Outubro} & \textbf{Novembro} & \textbf{Dezembro}\\\hline
Estudar SOA, OSGi e iPojo& \textbf{×} & \textbf{×} &   & \textbf{×}\\\hline
Implementar Interceptadores & \textbf{×} & \textbf{×} &  & \\\hline
Estudar DOSGi &   & \textbf{×} &   & \textbf{×}\\\hline
Implementar Seleção de Serviços &   & \textbf{×} & \textbf{×} &  \\\hline
Elaborar Monografia &   &   & \textbf{×} & \textbf{×}\\\hline
\end{supertabular}
\caption{Tabela das Atividades Programadas}
\end{table*}
\end{center}
}%

