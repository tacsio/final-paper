\paragraph{}
Service Oriented Architecture (SOA) has emerged as one of the preferred approaches for building systems.
With Goals aligned to the goals of the SOC paradigm, SOA enables the creation of new applications with greater consistency, speed and decrease in costs, all with excellent use of the legacy. Therefore, the construction of service-oriented applications tend to have greater agility, flexibility and reuse of pre-existing components. 
In this context, modules can consume services provided by different technologies, only identifying the required service and performing the binding, thus establishing a contract between the consumer and the service provider, defined by SLAs (Service Level Agreement). However, ensure that these contracts are properly carried out is an important and complex task for service providers.

This work aims to specify and build a mechanism for dynamic management of service providers, focused on prevention of possible breaches of contract depending on the context in which the service provided is performed.


\begin{keywords}
Service Oriented Architecture, Service Level Agreement, Dynamic Management
\end{keywords}
