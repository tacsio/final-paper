\paragraph{}
Arquitetura Orientada a Serviços, ou do inglês \textit{Service-Oriented Architecture} (SOA), tem surgido ao longo dos últimos anos como uma das abordagens preferidas para construção de sistemas. Com objetivos alinhados aos objetivos do paradigma SOC \textit{(Service-Oriented Computing)}, SOA possibilita a criação de novas aplicações com maior coerência, rapidez e diminuição nos custos, tudo isso com excelente aproveitamento do legado.
Deste modo, a construção de aplicações orientadas a serviços tende a ter maior agilidade, flexibilidade e reuso de componentes pré-existentes.
Neste contexto, módulos podem consumir serviços providos por tecnologias diferentes, apenas identificando o serviço requerido e realizando o \textit{binding}, estabelecendo assim um contrato entre o consumidor e o provedor do serviço, esse definido através de SLAs \textit{(Service Level Agreement)}. Porém, garantir que esses contratos são devidamente cumpridos é uma tarefa importante e complexa para provedores de serviços.

Tendo em vista esse cenário, o objetivo deste trabalho é especificar e construir um mecanismo de gerenciamento dinâmico de provedores de serviço, focado na prevenção de possíveis quebras de contrato em função do contexto em que o serviço provido é executado.

\begin{keywords}
Arquitetura Orientada a Serviços, Service Level Agreement, Gerenciamento Dinâmico
\end{keywords}
