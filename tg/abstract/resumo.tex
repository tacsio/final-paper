\paragraph{}
Arquitetura Orientada a Serviços, ou do inglês \textit{Service-Oriented Architecture} (SOA), tem surgido ao longo dos últimos anos como uma das abordagens preferidas para construção de sistemas. Com objetivos alinhados aos objetivos do paradigma SOC \textit{(Service-Oriented Computing)}, SOA possibilita a criação de novas aplicações com maior coerência, rapidez e diminuição nos custos, tudo isso com excelente aproveitamento do legado.
Deste modo, a contrução de aplicações orientadas a serviços tende a ter maior agilidade, flexibilidade e reuso de componentes pré-existentes.
Nesse contexto, módulos podem consumir serviços providos por tecnologias diferentes, apenas identificando o serviço requerido e realizando o \textit{binding}. Estabelecendo assim um contrato entre o consumidor e o provedor do serviço.
%TODO: Colocar novo objetivo

\begin{keywords}

\end{keywords}
