\chapter{Conceitos Básicos}
\label{ch:2}
Neste capítulo discutiremos alguns conceitos básicos necessários ao entendimento da solução proposta. Abordaremos inicialmente sobre orientação a serviços e a componentes, apresentando os pontos positivos e negativos de cada abordagem. 

Em seguida, discutiremos sobre gerencimento de recursos e serviços, sua importância no contexto de arquiteturas orientadas a serviços, e uma visão geral do ciclo de desenvolvimento PDCA. 

Por fim, apresentaremos uma visão geral de JMX, a tecnologia utilizada no trabalho para realizar o monitoramento dos recursos.

\section{Orientação a Serviços}
Orientação a Serviços, surge como um novo paradigma de desenvolvimento de aplicações distribuídas, promovendo reuso, baixo acoplamento e interoperabilidade do conceito de serviços como unidades que representam funcionalidades da aplicação~\cite{cervantes2005technical}~\cite{erl2008soa}. Estas funcionalidades são definidas através de contratos de serviço, nos quais o desenvolvimento da aplicação é baseado.

Assim, na orientação a serviços, as capacidades dos serviços são definidas através de uma \textit{Service Description} e a partir dessa descrição, provedores de serviço são descobertos por consumidores de serviço estabelecendo um contrato de serviço entre as partes. Deste modo, no mundo de serviços, temos diversas implementações distintas para um mesmo serviço onde o consumidor não está vinculado a nenhuma delas, podendo substituir o serviço consumido de acordo com suas necessidades, através da definição de novos contratos de serviço com base nessas descrições~\cite{cervantes2005technical}.

\subsection{SOA}
Alinhado a essas idéias, temos em SOA, um modelo de arquitetura baseada em serviços com suporte a descoberta dinâmica de serviços, uma vez que, o modelo baseia-se nos 3 elementos fundamentais da orientação a serviços.

\begin{itemize}
 \item \textbf{Provedor de Serviços} (\textit{Service Provider})
 \item \textbf{Consumidor de Serviços} (\textit{Service Requestor})
 \item \textbf{Registro de Serviços} (\textit{Service Registry})
\end{itemize}

A interação destes elementos compões o triângulo tradicional do modelo arquitetural, ver Figura \ref{fig:soatriangle}. 

O modelo de interação é relativamente simples. Provedores de serviço publicam suas capacidades através de \textit{Service Descriptions} no registro de serviços.
O consumidor do serviço executa consultas a um determinado serviço através do registro, essa consulta é feita com base da descrição do serviço. Caso exista um provedor que atenda as necessidades do consumidor, então o registro devolverá uma referência do provedor daquele serviço ao consumidor, possibilitando a realização do \textit{binding} entre eles. Então, o provedor disponibiliza um objeto que implementa a interface do serviço (\textit{Servant}~\cite{volter2005remoting}) finalizando a interação com o provedor. Quando a interação entre o consumidor e o \textit{servant} é finalizada, o mesmo é liberado para utilização em alguma outra requisição ou simplesmente destuído.



\section{Orientação a Componentes}
TODO

\section{Clico PDCA de Desenvolvimento}
TODO

\begin{figure}[htp]
\centering
\includegraphics[width=5cm]{chapters/intro/pdca_cycle.png}
\caption[Ciclo PDCA]{Ciclo PDCA de Desenvolvimento com Foco na Melhoria Contínua.}
\label{fig:pdca}
\end{figure}

\section{JMX}
\textit{Java Management Extensions} (JMX) é uma API que fornece uma maneira simples e padrão de gestão e monitoramento de recursos para a plataforma Java. Estes recursos podem ser aplicações, dispositivos, serviços ou a própria JVM (\textit{Java Virtual Machine}). São instrumentados por um ou mais \textit{Managed Beans}, ou simplesmente MBeans, responsáveis por adquirir, manipular ou enviar informações acerca destes recursos~\cite{lindfors2002jmx}.

Essa tecnologia foi desenvolvida através do JCP (\textit{Java Community Process}), em duas JSRs (\textit{Java Specification Request}) distintas. A JSR 3, \textit{Java Management Extensions Instrumentation and Agent Specification} e a JSR 160, \textit{Java Management Extensions Remote API}, ver Seção \ref{subsec:arch}.

JCP é o processo de desenvolvimento padrão de novas especificações para a plataforma Java, as chamadas JRSs. JSRs são documentos que descrevem as especificações e tecnologias propostas à plataforma.

A especificação de JMX define, além da arquitetura, padrões de projeto, API's e um conjunto de serviços de gerenciamento e monitoramento. Possibilitando o desenvolvimento de aplicações gerenciáveis local ou remotamente através do processo de instrumentação, onde atributos, configurações e capacidades da aplicação são expostos. Aumentando a robustez e extensibilidade da aplicação, uma vez que, é possível construir soluções de gerenciamento inteligentes, interoperáveis e independentes da infra-estrutura de gestão~\cite{jmx}.

\subsection{Arquitetura}
\label{subsec:arch}
A arquitetura JMX é definida em três níveis, ver Figura \ref{fig:arch_jmx}:

\begin{itemize}
 \item Nível de Instrumentação;
 \item Nível de Agente;
 \item Nível de Gerenciamento.
\end{itemize}

Como foi dito anteriormente, JMX foi definida segundo duas JSRs , JSR 3 e JSR 160. Os dois primeiros níveis da arquitetura (Instrumentação e Agente) foram definidos na JSR 3, enquanto o Nível de Gerenciamento foi definido na JSR 160. Isso mostra, de certa forma, o potencial de extensibilidade da tecnologia.

\begin{figure}[htp]
\centering
\includegraphics[width=12cm]{chapters/chapter4/arch_jmx.png}
\caption[Arquitetura JMX]{Arquitetura JMX.}
\label{fig:arch_jmx}
\end{figure}

\subsubsection{Nível de Intrumentação}
O nível de instrumentação é responsável por expor as funcionalidades e configurações das aplicações através da criação e registro de MBeans~\cite{jmx}. Estes MBeans coletam e manipulam informações dos recursos gerenciáveis, repassando-as aos agentes JMX do nível superior.

Exitem diferentes tipos de MBeans em JMX. Nas próximas seções descreveremos alguns dos mais importantes.

\paragraph{Standard MBean:}
\label{para:stardardmbean}
São os tipos mais simples de \textit{Managed Bean}. Interagem com os recursos gerenciáveis através da definição de uma interface de gerenciamento, que descreve os atributos e operações do MBean. Estas interfaces são definidas explicitamente e os atributos e métodos descobertos por meio de reflexão. Além disso, seguem a convenção do nome da classe acrescido do prefixo \verb MBean  e todos os seus atributos são acessados por métodos \textit{getters} e \textit{setters}. \textit{Standard Mbeans} possuem uma limitação quanto aos tipos de dados que podem ser utilizados, não permitindo o uso de tipos complexos definidos pelo usuário~\cite{mxbeans}.

Uma limitação da arquitetura JMX é que, não é permitido a implementação de duas interfaces de gerenciamento para a mesma MBean. Mesmo através de herança visando evitar a implementação explícita de duas interfaces, o agente JMX sempre irá utilizar a interface de gerenciamento mais próxima da classe. Uma alternativa a esse problema é estender a interface de gerenciamento.

\paragraph{MxBean:}
\label{para:mxbean}
Mxbean é uma versão melhorada do \textit{Standard MBean}, ele traz uma solução ao problema relacionado ao conjunto limitados de tipos de dados utilizáveis pela MBean comum, visto no parágrafo \ref{para:stardardmbean}.

Ele provê suporte à utilização de tipos de dados complexos, para representação de atributos do MBean, através de mapeamentos de tipos complexos para um tipo padrão \verb CompositeDataSupport ~\cite{mxbeans}.

A especificação de Mxbeans está presente na JSR 174 e este foi incluído na versão 6 de Java. Sendo assim, só pode ser executado nas versões superiores a J2SE 5.0~\cite{mxbeans}.

\paragraph{Dynamic MBean:}
\label{para:dynamicmbean}
\textit{Dynamic MBeans} se diferenciam de \textit{Standard MBeans} pelo fato de que estes descrevem seus atributos e operações através de uma interface genérica. Assim, as propriedades do MBean são descobertas em tempo de execução, já que, através dessa interface, os agentes JMX obtém a descrição da MBean por meio da classe \verb MBeanInfo ~\cite{jmx}. 

O diferencial é que no caso das \textit{Standard MBeans}, os metadados que descrevem a interface são obtidos por meio de reflexão, e na \textit{Dynamic MBean}, as classes de metadados são construídos por elas mesmas em tempo de execução.

Desde modo, o acesso ao atributos e operações em MBeans dinâmicas não é realizado diretamente, mas sim por meio de métodos genéricos, semelhante ao mecanismo utilizado pelo padrão de projeto \textit{Requestor}~\cite{volter2005remoting}. Esse método de acesso é ilustardo na figura \ref{fig:requestjmx}.

\begin{figure}[htp]
\centering
\subfloat[Invocação de Operação na MBean]{
\includegraphics[width=13cm]{chapters/chapter4/invoke.png}
}
\\
\subfloat[Acesso a atributo na MBean]{
\includegraphics[width=13cm]{chapters/chapter4/getValue.png}
}
\caption{Acesso a atributos e operações da MBean.}
\label{fig:requestjmx}
\end{figure}

Exitem outros tipos de MBeans dinâmicas, como Model MBean, Open MBean e AbtractDynamicMBean que basicamente adicionam novas features à Dynamic MBean tradicional. Não entraremos em detalhes pois está fora do escopo deste trabalho.

\subsubsection{Nível Agente}
O nível agente é responsável por gerenciar todas as Mbeans e através delas controlar diretamente todos os recursos, tornando-os acessíveis a aplicativos de gerenciamento remoto.

A gerência das MBeans é provida através do principal componente do agente JMX, o MBean Server. Além disso, agentes JMX incluem um conjunto de serviços de gerenciamento de MBeans e pelo menos um adaptador de comunicações ou conector para permitir o acesso ao agente por aplicativos de gerenciamento~\cite{jmx}.

Basicamente, um agente JMX consiste em um MBean Server, um conjunto de serviços básicos de gerência e pelo menos um adaptador de comunicação ou conector JMX.

\paragraph{MBean Server:}
O MBean Server é o principal componente do nível agente. É o intermediário entre o sistema de gerenciamento e os objetos gerenciáveis (recursos), uma vez que, é responsabilidade do MBean Server registrar e manipular as MBeans~\cite{jmx}.

Para registrar uma MBean no servidor, além do próprio objeto MBean, é necessário que seja registrado junto a MBean um ObjectName que identifica unicamente aquela MBean no servidor. Além de identificar a MBean, ele permite a seleção de objetos a partir de mascaras no formato [domínio]:[par chave/valor]. \\
\verb org.meudominio:local=UFPE,nome=MBean \\

A manipulação de MBeans é feita através do MBean Server que provê um conjunto de operações comuns aos diferentes tipos de MBeans, uma vez que, os atributos e operações são descobertos através de classes de metadados comuns, ver parágrafo \ref{para:dynamicmbean}.

\subsubsection{Nível de Gerenciamento}
O nível de gerenciamento é o nível mais alto da arquitetura JMX, este define um mecanismo baseado em adaptadores e conectores que tornam os agente JMX acessíveis a aplicativos de gerenciamento remoto fora da JVM em que o agente encontra-se~\cite{jmx}.

Define, de certa forma, uma interface de acesso a agentes JMX para o mundo externo, através de protocolos proprietários ou existentes (\textit{e.g.} SNMP - \textit{Simple Network Manager Protocol}, um protocolo padrão para gerenciamento de componentes de rede)~\cite{douglas2005essential}.

Essa interface é garantida através de conectores e adaptadores que disponibilizam o acesos aos agentes JMX para diferentes tecnologia de gerenciamento.

A principal diferença entre conectores e adaptadores é que adaptadores surgem como uma solução de integração a sistemas que não dão suporte direto a JMX, por exemplo, sistemas de gerenciamento baseados em SNMP ou HTTP como Zabbix~\cite{zabbix} ou Nagios~\cite{nagios}, fornecendo uma visão de todos os MBeans através de um determinado protocolo. 

Já conectores, fornecem uma interface de gerenciamento remoto transparente e independente de protocolo~\cite{jmx}.


\subsection{Mecanismo Notificações}
Além dos três níveis de arquitetura, JMX provê um mecanismo de notificação que permite que aplicações ou MBeans recebam eventos ou notificações com informações sobre o estado dos recursos gerenciados, podendo gerar estatísticas ou mesmo processar eventos relacionados ao funcionamento dos recursos~\cite{lindfors2002jmx}.

O modelo de notificação JMX é semelhante ao mecanismo de notificações de java, que define \textit{object listers} (recebem as notificações) registados a MBean que envia as notificações \textit{(broadcaster MBeans)}. A \textit{broadcaster MBeans} envia eventos de notificação a todos os \textit{listeners} registrados~\cite{lindfors2002jmx}.

\subsubsection{Componentes do Mecanismo de Notificações}

O mecanismo de notificações JMX possui os seguintes componentes:

\begin{center}
\begin{table*}[h]
\begin{supertabular}[]{|l|l|}
\hline
\textbf{Componente} & \textbf{Descrição}\\\hline
Notification & Representa um tipo genérico de notificação\\\hline
NotificationListener & Interface que permite o recebimento de notificações\\\hline
\multirow{2}{*}{NotificationFilter} & Interface que permite a filtragem de notificações. Assim, apenas\\ 
& notificações relevantes ao Listener são recebidas\\\hline
\multirow{2}{*}{NotificationBroradcaster} & Interface que permite que notificações sejam enviadas aos\\
& listeners registrados \\\hline
\end{supertabular}
\caption{Componentes do Mecanismo de Notificação}
\end{table*}
\end{center}

\subsubsection{Modelo de Notificação JMX}
O modelo de notificações JMX segue os seguintes passos, demonstrados na Figura \ref{fig:notifyjmx}

\paragraph{Funcionamento} 
\begin{enumerate}
\item MBean consumidor registra-se através da interface \textit{NotificationBroadcaster}

\item MBean gerador emite uma notificação

\item A notificação é enviada aos MBeans registrados

\item O MBean consumidor recebe e filtra as notificações

\item As notificações não descartadas são tratadas
\end{enumerate}

\begin{figure}[htp]
\centering
\includegraphics[width=13cm]{chapters/chapter4/notification_model.png}
\caption[Modelo de notificação JMX]{Modelo de notificação JMX.}
\label{fig:notifyjmx}
\end{figure}

\section{Considerações Finais}