\chapter{Conclusão e Trabalhos Futuros}
\label{ch:5}
%TODO introduzir capitulo

\section{Contribuições}

%FIXME organizar contribuicoes
- suporte a utilização de diferentes politicas de criação de instancias em um ambiente OSGi distribuído. (Não suportado pelas políticas padrões do iPOJO.
- suporte a troca dinâmica de politicas de criação de instancias.


\section{Trabalhos Futuros}
Como pudemos perceber, toda a definição das políticas de gerenciamento de instâncias é feita em tempo de \textit{deployment}. 

Assim, uma possível extensão a este trabalho, seria incorporar um componente inteligente que definisse qual a política que melhor atende ao contexto da aplicação, dentre as possíveis, em tempo de execução.

Outra possibilidade de extensão interessante, seria embutir a capacidade de predição de possíveis quebras de contrato, com base na análise da variação da utilização dos recursos. Para isso, seria necessário um estudo mais aprofundado dos padrões existentes de uso de recursos para cada caso específico. Uma solução baseada em redes neurais, aplicação de técnicas de mineração de dados ou mesmo armazenar tais dados em bases consolidadas para análises mais elaboradas realizadas por ferramentas OLAP, podem ser uma extensão interessante deste trabalho.